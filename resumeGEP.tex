%%%%%%%%%%%%%%%%%%%%%%%%%%%%%%%%%%%%%%%%%
% Medium Length Professional CV
% LaTeX Template
% Version 2.0 (8/5/13)
%
% This template has been downloaded from:
% http://www.LaTeXTemplates.com
%
% Original author:
% Trey Hunner (http://www.treyhunner.com/)
%
% Important note:
% This template requires the resume.cls file to be in the same directory as the
% .tex file. The resume.cls file provides the resume style used for structuring the
% document.
%
%%%%%%%%%%%%%%%%%%%%%%%%%%%%%%%%%%%%%%%%%

%----------------------------------------------------------------------------------------
%   PACKAGES AND OTHER DOCUMENT CONFIGURATIONS
%----------------------------------------------------------------------------------------

\documentclass{resume} % Use the custom resume.cls style

\usepackage{hyperref}
% document margins
\usepackage[left=0.75in,top=0.6in,right=0.75in,bottom=0.6in]{geometry}

% name
\name{Deric Pang}
% address
\address{4233 9th Ave NE \\ Seattle, WA 98105}
% github
\address{{\href{https://github.com/pderichai}{https://github.com/pderichai}}}
% phone number and email
\address{(702)~$\cdot$~606~$\cdot$~5830 \\ pderichai@gmail.com}

\begin{document}

%----------------------------------------------------------------------------------------
%   EDUCATION SECTION
%----------------------------------------------------------------------------------------

\begin{rSection}{Education}

  {\bf University of Washington, Seattle} \hfill {September 2014 - Present} \\
  B.S. in Computer Science \& Engineering (Sophomore) \\
  Dean's List \\
  Overall GPA: 3.86/4 \\
  Coursework: Computer Programming I \& II, Foundations of Computing, Hardware Software Interface, Software Design and Implementation

\end{rSection}

%----------------------------------------------------------------------------------------
%   WORK EXPERIENCE SECTION
%----------------------------------------------------------------------------------------

\begin{rSection}{Experience}

  \begin{rSubsection}{Undergraduate Researcher}
    {March 2015 -- Present}{University of Washington Computer Science \& Engineering}{Seattle, WA}
  \item Worked as part of the Programming Languages and Software Engineering
    group.
  \item Wrote approximately 3000 lines of code in the summer of 2015.
  \item Studied patch minimization and delta debugging---currently in the process of authoring a paper.
  \end{rSubsection}

  \begin{rSubsection}{Front End Developer}{September 2015}{AT\&T Mobile App Hackathon}{Seattle, WA}
  \item Worked to create \emph{Weelz}, a crowdsourced mobile bike hazard notification android application.
  \end{rSubsection}

  \begin{rSubsection}{Chamber Music Club Officer}{June 2015 -- Present}{Project and Event Planner}{University of Washington}
  \item Arranged members into chamber groups and planned concerts quarterly.
  \end{rSubsection}

\end{rSection}

%----------------------------------------------------------------------------------------
%   PROJECTS SECTION
%----------------------------------------------------------------------------------------

\begin{rSection}{Projects}

  \begin{rSubsection}{Patch Minimizer}{March 2015 -- Present}{}{}
  \item Automatically isolates buggy code between two versions of a program.
  \item Approximately 2000 lines of code.
  \item Can be used with any language or project.
  \end{rSubsection}

  \begin{rSubsection}{\href{https://github.com/pderichai/diff-utils}{Diff-Utils}}{August 2015}{}{}
  \item A diff utility that allows for the syntactically correct manipulation and
    modification of unified diff files.
  \item Created for use in patch minimization research.
  \end{rSubsection}

  \begin{rSubsection}{Weelz}{September 2015}{}{}
  \item A crowd sourced bike hazard notification application.
  \item Worked on front end android application development.
  \end{rSubsection}

\end{rSection}

%----------------------------------------------------------------------------------------
%   TECHNICAL STRENGTHS SECTION (Renamed)
%----------------------------------------------------------------------------------------

\begin{rSection}{Skills}

  \begin{tabular}{ @{} >{\bfseries}l @{\hspace{6ex}} l }
    Proficient In & Java, Linux, Bash \\ Experience With & C, HTML \& CSS,
    JavaScript, Android SDK, Google Maps API \\ Tools & Git, Emacs, Vim, Defects4J
  \end{tabular}

\end{rSection}

%----------------------------------------------------------------------------------------
\begin{rSection}{Essay Questions}
  \begin{rSubsection}{Question 1}{}{}{}
    \item My passion for computer science was solidified by the research I started towards the
    end of my freshman year. Once I started working in the Programming Languages and
    Software Engineering group, I knew that I wanted to be a software engineer
    for the rest of my life. The work I did never really felt like work---it was as if somebody
    was paying me to have fun. The joy I experience from overcoming challenging problems
    with my knowledge of coding is like no other.

    My goal for this upcoming summer is to learn as much as I can. As a Google intern,
    I would get to work alongside the best
    engineers in the industry and influence some of the most widely used products on the
    planet. This is the kind of experience I am looking for in order to become more
    knowledgeable and capable. Before college, I had not done much coding, and I think this forced
    me to be an effective learner. In less than
    five months, I went from barely being able to code to working alongside grad
    students and professors in research. I know that if I have the opportunity to intern
    at Google, I would be able to learn a tremendous amount.
  \end{rSubsection}

   \begin{rSubsection}{Question 2}{}{}{}
    \item I am so glad you guys asked this question because I love talking about my work!

    For the research paper I'm currently working on, I built a patch minimizer in Java.
    In order to build the patch minimizer, I had to overcome many technical challenges,
    one of which was the syntactically correct manipulation of unified diff files.

    Unified diff files are tricky to change. Consider changes that are to be \textit{removed}
    from source code after a patch is applied. In order to "remove" these changes from a patch, it's necessary 
    to only remove the '-' sign at the beginning of the line so that the affected lines remain
    as context for the diff. However,
    if the context information of the patch isn't changed, it's possible that removing the
    '-' sign from the patch will render it inapplicable.

    As you can see, this problem turned out to be quite tricky, and was sort of a side project
    on its own. At first, I tried to use various libraries I found online to solve the problem,
    but none of them were adequate. In the end, I wrote my own program to do this---one that I
    am quite proud of. Check it out on my GitHub!
   \end{rSubsection}

   \begin{rSubsection}{Question 3}{}{}{}
   \item Diversity in the workplace is extremely important because it allows for
     ingenuity. A team composed of people who are very similar makes it tough 
     to innovate---many angles of a difficult problem must be explored before a new 
     solution can be found.

     I like to think that I represent diversity in the tech industry well. In addition
     to studying computer science, I have competed in violin competitions across the
     nation and led my high school tennis team to two state championships. I strongly
     believe that being well versed makes me a better programmer by giving me the ability to
     approach problems differently and by instilling in me values that are essential
     to engineering.

     Many discussions have been going on recently about the lack of diversity in the
     tech industry. These discussions are important for the very reasons I just
     listed---the tech industry needs not only good engineers, but also good engineers
     that come from different backgrounds and have different ways of thinking. I think
     the best way to increase diversity in the tech industry is to encourage underrepresented
     groups to study computer science. Thankfully, this push for diversity is already happening at many
     schools and universities.

   \end{rSubsection}
  
\end{rSection}
        
%----------------------------------------------------------------------------------------
%   EXAMPLE SECTION
%----------------------------------------------------------------------------------------

%\begin{rSection}{Section Name}

%Section content\ldots

%\end{rSection}

%----------------------------------------------------------------------------------------

\end{document}
