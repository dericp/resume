%%%%%%%%%%%%%%%%%%%%%%%%%%%%%%%%%%%%%%%%%
% Medium Length Professional CV
% LaTeX Template
% Version 2.0 (8/5/13)
%
% This template has been downloaded from:
% http://www.LaTeXTemplates.com
%
% Original author:
% Trey Hunner (http://www.treyhunner.com/)
%
% Important note:
% This template requires the resume.cls file to be in the same directory as the
% .tex file. The resume.cls file provides the resume style used for structuring the
% document.
%
%%%%%%%%%%%%%%%%%%%%%%%%%%%%%%%%%%%%%%%%%

%----------------------------------------------------------------------------------------
%   PACKAGES AND OTHER DOCUMENT CONFIGURATIONS
%----------------------------------------------------------------------------------------

\documentclass{resume} % Use the custom resume.cls style

\usepackage{hyperref}
% document margins
\usepackage[left=0.75in,top=0.6in,right=0.75in,bottom=0.6in]{geometry}

% name
\name{Deric Pang}

% website
\address{{\href{https://homes.cs.washington.edu/~dericp}{https://homes.cs.washington.edu/\textasciitilde{}dericp}}}
% github
\address{{\href{https://github.com/dericp}{https://github.com/dericp}}}
% email and phone number
\address{\href{mailto:dericp@cs.washington.edu}{dericp@cs.washington.edu}}


\begin{document}

%----------------------------------------------------------------------------------------
%   TECHNICAL STRENGTHS SECTION (Renamed)
%----------------------------------------------------------------------------------------

\begin{rSection}{Skills Summary}

	%\begin{rQualifications}
	%	\item Built a production system which integrated with AWS to validate credit card and bank account numbers.
	%	\item Built speech recognition systems using machine learning and language processing techniques.
	%	\item Developed an \href{https://github.com/dericp/patch-minimization}{automated bug finder} and co-authored \href{https://homes.cs.washington.edu/~dericp/resources/fault-localization-tr160803.pdf}{\emph{Evaluating \& improving fault localization techniques}}.
	%\end{rQualifications}

\begin{tabular}{ @{} >{\bfseries}l @{\hspace{3ex}} l }
	Languages: & Python, Java, C, C++, Shell, Scala, HTML \& CSS, JavaScript, PHP, \LaTeX
	\\ Tech/Tools: & TensorFlow, MXNet, PyTorch, AWS, Git, Ant, Gradle, Kaldi
\end{tabular}

\end{rSection}


%----------------------------------------------------------------------------------------
%   EDUCATION SECTION
%----------------------------------------------------------------------------------------

\begin{rSection}{Education}

  {\href{https://www.cs.washington.edu/}{\bf University of Washington, Seattle}} \hfill {Sept. 2014 -- Present} \\
  B.S. \& M.S. in Computer Science \\
  Paul G. Allen School of Computer Science \& Engineering \\
  Dean's List every quarter \\
  Overall GPA: 3.77/4.00

  {\href{https://www.inf.ethz.ch/}{\bf Swiss Federal Institute of Technology in Z\"{u}rich (ETH Z\"{u}rich)}} \hfill {Fall 2016} \\
  University of Washington Computer Science \& Engineering Direct Exchange  \\
  Took graduate courses in computer science: Data Mining, Information Retrieval

\end{rSection}

%----------------------------------------------------------------------------------------
%   WORK EXPERIENCE SECTION
%----------------------------------------------------------------------------------------

\begin{rSection}{Experience}

  \begin{rSubsection}{Alexa Machine Learning --- {\href{https://www.amazon.com/}{Amazon}}}{June 2017 -- Sept. 2017}{Software Development Engineering Intern}{Seattle, WA}
    \item Worked on Amazon's internal deep learning framework which was specialized for automatic
    speech recognition.
  \end{rSubsection}

  \begin{rSubsection}{\href{https://uwplse.org/}{Programming Languages and Software Engineering Lab}}
    {Mar. 2015 -- Present}{Undergraduate Researcher, advised by \href{https://homes.cs.washington.edu/~mernst/}{Michael Ernst}, \href{https://www.cs.washington.edu/people/faculty/lsz}{Luke Zettlemoyer}, and \href{https://people.cs.umass.edu/~rjust/}{Ren{\'e} Just}}{University of Washington}
  \item Working on the \href{https://github.com/TellinaTool}{Tellina} project to generate bash commands from natural language.
  \item Built an \href{https://github.com/dericp/patch-minimization}{automatic bug finder} using patch minimization and delta debugging techniques.
  \item Co-authored \href{https://homes.cs.washington.edu/~dericp/resources/fault-localization-tr160803.pdf}{\emph{Evaluating \& improving fault localization techniques}} --- accepted to \href{http://icse2017.gatech.edu/}{ICSE 2017}.
  \end{rSubsection}
  
  \begin{rSubsection}{\href{http://www.marchex.com/}{Marchex}}
    {June 2016 -- Sept. 2016}{Software Engineering/Research Intern}{Seattle, WA}
  \item Built a speech recognition system using deep learning techniques to transcribe phone calls.
  \item Trained a neural network based on the Deep Speech 2 architecture.
  \item Transcribed Australian English with the Kaldi automatic speech recognition toolkit.
  \end{rSubsection}

  \begin{rSubsection}{\href{https://www.amazon.com/}{Amazon}}
    {Mar. 2016 -- June 2016}{Software Development Engineering Intern}{Seattle, WA}
  \item Developed business critical software in Amazon Payment Services to help validate payment instruments like credit card and bank account numbers.
  \item Integrated with AWS technologies such as AWS SWF, Lambda, S3, DynamoDB, SQS, and SNS.
  \end{rSubsection}
  
  \begin{rSubsection}{\href{https://courses.cs.washington.edu/courses/cse446/17sp/}{Machine Learning} | \href{https://courses.cs.washington.edu/courses/cse331/16wi/}{Software Design \& Implementation}}
    {Winter 2016 -- Present}{Teaching Assistant for CSE 446 and CSE 331}{University of Washington}
  \item Planned and delivered lectures during weekly recitations.
  \item Graded and provided feedback for weekly programming projects.
  \item Met weekly with the lecturing professor to discuss teaching, grading, and course progress.
  \end{rSubsection}
  
\end{rSection}


%----------------------------------------------------------------------------------------
%   EXAMPLE SECTION
%----------------------------------------------------------------------------------------

%\begin{rSection}{Section Name}

%Section content\ldots

%\end{rSection}

%----------------------------------------------------------------------------------------

\end{document}
